%% Glosario
\begin{glosario}
    \item[Aminoácido:] Unidad molecular básica que constituye las proteínas. Existen 20 aminoácidos estándar que se combinan en diferentes secuencias para formar proteínas con distintas funciones.

    \item[Angstrom (\AA):] Unidad de longitud equivalente a $10^{-10}$ metros, comúnmente utilizada para medir distancias atómicas y moleculares en biología estructural.

    \item[Backtracking:] Proceso de recorrer la matriz de programación dinámica en sentido inverso para reconstruir las subsecuencias comunes más largas a partir de la información almacenada durante el cálculo de la matriz LCS.

    \item[Biopython:] Biblioteca de Python que proporciona herramientas para trabajar con datos biológicos, incluyendo análisis de secuencias y estructuras de proteínas.

    \item[Bio.PDB:] Módulo de Biopython especializado en el análisis de estructuras proteicas tridimensionales, capaz de parsear archivos PDB y realizar búsquedas espaciales.

    \item[Eficiencia (Efficiency):] Métrica de rendimiento paralelo definida como $\text{Eficiencia} = \frac{\text{Speedup}}{N} \times 100\%$, donde $N$ es el número de workers. Indica qué tan efectivamente se utilizan los recursos computacionales.

    \item[Goroutine:] Hilo de ejecución liviano en el lenguaje Go que permite ejecutar funciones de forma concurrente con bajo overhead de memoria y gestión simplificada.

    \item[Grano grueso (Coarse-grained):] Estrategia de paralelización donde cada unidad de trabajo es grande y toma un tiempo significativo (milisegundos o más), minimizando el overhead de sincronización.

    \item[Grano fino (Fine-grained):] Estrategia de paralelización donde cada unidad de trabajo es pequeña (microsegundos), requiriendo alta sincronización y siendo más susceptible al overhead.

    \item[LCS (Longest Common Subsequence):] Subsecuencia Común Más Larga. Algoritmo de programación dinámica que encuentra la secuencia más larga de elementos que aparecen en el mismo orden en dos secuencias, no necesariamente de forma consecutiva.

    \item[Ligando:] Molécula que se une a una proteína específica, típicamente en un sitio activo, formando un complejo que puede tener relevancia biológica o farmacológica.

    \item[Mutex:] Mecanismo de sincronización (mutual exclusion) que permite a múltiples goroutines acceder de forma segura a recursos compartidos, evitando condiciones de carrera.

    \item[NeighborSearch:] Algoritmo de búsqueda espacial implementado en Bio.PDB que identifica átomos o residuos cercanos a un punto dado dentro de un radio especificado.

    \item[PA Line (PROSITE):] Notación estándar de PROSITE para representar patrones de secuencias proteicas, utilizando caracteres en mayúsculas para posiciones conservadas y x(n) para representar huecos de longitud variable.

    \item[Paralelización:] Técnica de programación que distribuye el trabajo computacional entre múltiples unidades de procesamiento (núcleos, procesadores) para reducir el tiempo de ejecución.

    \item[PDB (Protein Data Bank):] Base de datos pública que almacena información estructural tridimensional de proteínas y ácidos nucleicos, proporcionando archivos en formato mmCIF o PDB.

    \item[Programación dinámica:] Técnica algorítmica que descompone problemas complejos en subproblemas más simples, almacenando sus resultados para evitar cálculos redundantes.

    \item[Proteína:] Macromolécula biológica compuesta por cadenas de aminoácidos que desempeña funciones estructurales, catalíticas, regulatorias y de transporte en los organismos vivos.

    \item[PROSITE:] Base de datos de patrones y perfiles de proteínas que documenta sitios biológicamente significativos, utilizando notaciones estándar como PA Line.

    \item[Residuo:] En el contexto de proteínas, un aminoácido específico en una posición determinada de la cadena polipeptídica, identificado por su número de secuencia y código de inserción.

    \item[Speedup:] Métrica de rendimiento paralelo definida como $S = \frac{T_{\text{secuencial}}}{T_{\text{paralelo}}}$, que indica cuántas veces más rápido es la versión paralela respecto a la secuencial.

    \item[WaitGroup:] Primitiva de sincronización del paquete sync de Go que permite esperar a que un conjunto de goroutines termine su ejecución antes de continuar.

    \item[Zinc finger:] Dominio estructural de proteínas que coordina iones de zinc mediante residuos de cisteína e histidina, común en factores de transcripción que se unen al ADN.
\end{glosario}
