\chapter{Contexto}

Este capítulo establece el marco conceptual y biológico necesario para comprender el problema de descubrimiento de patrones en interacciones proteína-ligando. Se introducen los conceptos fundamentales de proteínas, ligandos y su representación como cadenas de caracteres, seguido de una definición formal de los patrones proteína-ligando y su importancia en bioinformática. Se presenta la notación PA Line de PROSITE, utilizada para expresar estos patrones de forma estandarizada, y se revisan los métodos existentes para su descubrimiento, incluyendo el trabajo previo que motiva este proyecto. Este contexto biológico y técnico es esencial para entender las decisiones de diseño del algoritmo desarrollado y la elección del enfoque basado en Longest Common Subsequence (LCS) para abordar el problema.

\section{Proteínas, ligandos y cadenas}

Las definiciones formales de algunos conceptos importantes son:
\begin{itemize}
    \item Aminoácidos: Son moléculas que se combinan para formar proteínas.\cite{defAminoacido}
    \item Proteínas: Son moléculas grandes y complejas que desempeñan muchas funciones críticas en el cuerpo. Realizan la mayor parte del trabajo en las células y son necesarias para la estructura, función y regulación de los tejidos y órganos del cuerpo.\cite{defProteina}
    \item Ligandos: Es una sustancia (usualmente una molécula pequeña) que forma un complejo con una biomolécula. En un sentido más estricto, es una molécula que envía una señal al unirse al centro activo de una proteína. \cite{defLigando} Es decir, un ligando es una molécula que se une a una proteína, alterando su estructura y función.
\end{itemize}

Hay 20 tipos diferentes de aminoácidos que se pueden combinar para formar una proteína. La secuencia de aminoácidos determina la estructura tridimensional única de cada proteína y su función específica \cite{defProteina}.

El Cuadro \ref{tab:aminos} contiene la lista de aminoácidos estándar.

\begin{table}[H]
\centering
\begin{tabular}{|l|c|c|}
\hline
\textbf{Nombre} & \textbf{Abreviatura} & \textbf{Símbolo} \\
\hline
Alanina & Ala & A \\
Arginina & Arg & R \\
Asparagina & Asn & N \\
Aspártico & Asp & D \\
Cisteína & Cys & C \\
Fenilalanina & Phe & F \\
Glicina & Gly & G \\
Glutámico & Glu & E \\
Glutamina & Gln & Q \\
Histidina & His & H \\
Isoleucina & Ile & I \\
Leucina & Leu & L \\
Lisina & Lys & K \\
Metionina & Met & M \\
Prolina & Pro & P \\
Serina & Ser & S \\
Tirosina & Tyr & Y \\
Treonina & Thr & T \\
Triptófano & Trp & W \\
Valina & Val & V \\
\hline
\end{tabular}
\caption{Lista de aminoácidos estándar.}
\label{tab:aminos}
\end{table}

Un ejemplo de una secuencia de proteína, correspondiente a la proteína 5PNQ \cite{prote5PNQ}, es el siguiente:

\begin{verbatim}
MHHHHHHSSGVDLGTENLYFQSMETMKSKANCAQNPNCNIMIFHPTKEEFNDFDKYIAYMESQG
AHRAGLAKIIPPKEWKARETYDNISEILIATPLQQVASGRAGVFTQYHKKKKAMTVGEYRHLAN
SKKYQTPPHQNFEDLERKYWKNRIYNSPIYGADISGSLFDENTKQWNLGHLGTIQDLLEKECGV
VIEGVNTPYLYFGMWKTTFAWHTEDMDLYSINYLHLGEPKTWYVVPPEHGQRLERLARELFPGS
SRGCGAFLRHKVALISPTVLKENGIPFNRITQEAGEFMVTFPYGYHAGFNHGFNCAEAINFATP
RWIDYGKMASQCSCGEARVTFSMDAFVRILQPERYDLWKRGQDR
\end{verbatim}

Adicionalmente y para facilitar el entendimiento estableceremos el siguiente vocabulario para referirnos a cada parte de la proteína que se obtiene durante el algoritmo:
\begin{itemize}
    \item Secuencia: Para referirnos a una proteína completa.
    \item Subsecuencia de interés: Parte de la proteína que interactúa con un ligando y recortamos.
\end{itemize}


\section{Patrones Proteína-Ligando}
\label{sec:Patrones Proteína-Ligando}

Un patrón proteína-ligando es una secuencia específica de aminoácidos alrededor de un ligando que se repite en múltiples proteínas. Estos patrones representan regiones funcionales donde ocurre la interacción entre la proteína y la molécula del ligando, lo cual es esencial para la actividad biológica de la proteína. La identificación y análisis de estos patrones permiten comprender mejor cómo las proteínas llevan a cabo sus funciones.

La detección de estos patrones es fundamental en diversas áreas, como el diseño de fármacos, donde se busca identificar moléculas que puedan unirse eficazmente a una proteína objetivo para modificar su actividad. También es crucial en la ingeniería de proteínas, donde modificar o diseñar nuevas proteínas con funciones específicas requiere un entendimiento detallado de los patrones de interacción proteína-ligando.

Para representar estos patrones en secuencias proteicas, se utilizan herramientas y notaciones especializadas como la PA Line de PROSITE\cite{manualpaline}, que permite describir patrones complejos de manera concisa y estandarizada. Esta notación facilita la comparación y el análisis de secuencias, permitiendo a los investigadores identificar rápidamente regiones de interés y predecir posibles interacciones funcionales.


\section{Notación PA Line de Prosite}

Para representar patrones en secuencias proteicas, se utiliza la notación PA Line de PROSITE\cite{manualpaline}, una herramienta ampliamente reconocida en bioinformática.

La notación PA Line es un formato estandarizado que permite describir de manera concisa y precisa los patrones de secuencias. Esta notación utiliza símbolos y reglas específicas para representar aminoácidos individuales, opciones alternativas en una posición dada, repeticiones y rangos de longitud. A continuación, se describen algunos elementos clave de esta notación:\cite{manualpaline}:
\begin{itemize}
    \item \textbf{Código IUPAC}: Se utilizan los códigos de una letra estándar para los amino-ácidos.
    \item \textbf{Posición indefinida}: Se usa el símbolo \texttt{x} para representar cualquier aminoáci-do en una posición dada.
    \item \textbf{Ambigüedades}: Se indican mediante corchetes \texttt{[]} enumerando los aminoáci-dos permitidos en una posición. Ejemplo: \texttt{[ALT]} significa Ala, Leu o Thr. $($por las letras A, L y T$)$
    \item \textbf{Exclusión de aminoácidos}: Se indica mediante llaves \texttt{\{\}} enumerando los aminoáci-dos no permitidos en una posición. Ejemplo: \texttt{\{AM\}} significa cualquier aminoácido excepto Ala o Met.
    \item \textbf{Repetición}: Se denota usando un número o un rango entre paréntesis. Ejemplo: \texttt{x(3)} representa tres posiciones consecutivas ocupadas por cualquier aminoá-cido, y \texttt{x(2,4)} indica de dos a cuatro posiciones consecutivas de cualquier aminoácido.
    \item \textbf{Restricciones en los extremos}: Se utiliza el símbolo \texttt{<} para restringir un patrón al extremo N-terminal y \texttt{>} para el extremo C-terminal de una secuencia.
    \item \textbf{Patrón finalizado}: Un patrón siempre termina con un punto (\texttt{.}).
    \item \textbf{Ejemplo de patrón}: \texttt{PA [AC]-x-V-x(4)-\{ED\}.} se traduce como: Ala o Cys, seguido de cualquier aminoácido, seguido de Val, seguido de cualquier cuatro aminoácidos, seguido de cualquier aminoácido excepto Glu o Asp.
\end{itemize}


\section{Métodos para descubrir patrones}


La búsqueda de patrones en secuencias proteicas es una tarea crucial en bioinformática, ya que permite identificar regiones conservadas que pueden ser funcionalmente relevantes, como sitios de unión a ligandos. A continuación, se describen algunos métodos y herramientas clave, incluyendo Pratt y su versión web, y se discute su utilidad para la identificación de patrones proteína-ligando.

\subsection{Pratt}


Pratt  es una herramienta ampliamente utilizada para la identificación de patrones conservados en conjuntos de secuencias proteicas. Mediante la búsqueda de patrones comunes en un conjunto de secuencias.

Pratt utiliza algoritmos que consideran la variabilidad en las secuencias, permitiendo identificar patrones que no necesariamente están perfectamente conservados, pero que presentan similitudes significativas. Esto es especialmente útil en el análisis de familias de proteínas donde la conservación es parcial debido a divergencias evolutivas.

Algunas características clave de Pratt incluyen:

\begin{itemize} 
    \item \textbf{Flexibilidad en la definición de patrones}: Permite especificar parámetros como la longitud mínima y máxima de los patrones, el número mínimo de secuencias que deben contener el patrón, y la posibilidad de incluir posiciones variables o ambiguas. 
    \item \textbf{Incorporación de gaps}: Puede manejar espacios en los patrones, lo que es útil cuando los motivos funcionales están separados por regiones variables. 
     
    \item \textbf{Salida en formato PROSITE}: Los patrones se presentan en la notación PA Line de PROSITE, facilitando su interpretación y uso en otras herramientas bioinformáticas. \end{itemize}


\subsection{Versión Web de Pratt}


La versión web de PRATT está disponible a través del servidor ExPASy, que es parte del Swiss Institute of Bioinformatics. Esta interfaz web permite a los usuarios ejecutar Pratt sin necesidad de instalar software localmente, facilitando su acceso y uso.

Además, el Protein Data Bank (PDB) ofrece recursos para el análisis de estructuras proteicas, y aunque no proporciona directamente una versión web de Pratt, las secuencias y estructuras disponibles en el PDB pueden ser utilizadas como entrada para Pratt. Los usuarios pueden extraer secuencias de proteínas del PDB y utilizarlas en Pratt para identificar patrones conservados.

Si bien Pratt es una herramienta potente para identificar patrones conservados en secuencias proteicas, no está específicamente diseñada para identificar patrones de interacción proteína-ligando. Los patrones descubiertos por Pratt corresponden a secuencias de aminoácidos que están conservadas entre diferentes proteínas, lo que puede incluir sitios de unión a ligandos si estos sitios están conservados.

La identificación de patrones proteína-ligando requiere información sobre las interacciones específicas entre los aminoácidos de la proteína y el ligando. Esto generalmente implica el análisis de datos estructurales 3D, como los disponibles en el PDB, y el uso de herramientas especializadas que consideran la geometría y las propiedades químicas de la interacción.


\subsection{Método base para descubrir patrones}
\label{algoritmooriginal}

Este trabajo se inspira en una propuesta secuencial presentada recientemente \cite{francisco}, que emplea un método exhaustivo de generación de subsecuencias para identificar patrones. El algoritmo original opera mediante tres etapas principales:

\textbf{Paso 1: Generación exhaustiva mediante máscaras de bits.}
Para cada secuencia de aminoácidos de longitud $n$, se generan todas las posibles subsecuencias aplicando una máscara de bits. Por ejemplo, para la secuencia \texttt{ABC}, se utilizan las máscaras \texttt{000}, \texttt{001}, \texttt{010}, \texttt{011}, \texttt{100}, \texttt{101}, \texttt{110}, \texttt{111}, generando las subsecuencias \texttt{ABC}, \texttt{ABx}, \texttt{AxC}, \texttt{Axx}, \texttt{xBC}, \texttt{xBx}, \texttt{xxC}, \texttt{xxx}, donde \texttt{x} representa un hueco (gap).

\textbf{Paso 2: Comparación exhaustiva.}
Se comparan todas las subsecuencias generadas entre sí para identificar coincidencias y alineamientos comunes entre las diferentes secuencias de aminoácidos.

\textbf{Paso 3: Formateo y asignación de puntuación.}
Cada patrón identificado se traduce a la notación PA Line, colapsando los huecos consecutivos en expresiones \texttt{x(n)} y manteniendo en mayúsculas los aminoácidos conservados. Finalmente, se calcula una puntuación para cada patrón, dando prioridad a aquellos con mayor cantidad de aminoácidos conservados (mayúsculas), permitiendo ordenar las alternativas.

\textbf{Ejemplo}

Consideremos las cadenas \texttt{ABCD} y \texttt{AHCD}. El método exhaustivo generaría todas las posibles subsecuencias de ambas cadenas y compararía cada par para encontrar coincidencias. Entre los patrones resultantes se encontraría:

\begin{itemize}
    \item \texttt{A-x(1)-C-D}
\end{itemize}

Este patrón captura la coincidencia de \texttt{A}, \texttt{C} y \texttt{D}, con un hueco correspondiente a la diferencia entre \texttt{B} y \texttt{H}.

\vspace{1em}

\textbf{Limitaciones y modificaciones.}
El método original no consideraba los aminoácidos interactuantes con el ligando, procesando secuencias completas sin discriminar qué residuos son biológicamente relevantes para la interacción. Por esta razón, se realizó una primera modificación al algoritmo para incorporar únicamente los aminoácidos que interactúan directamente con cada ligando, reduciendo así el espacio de búsqueda a los segmentos funcionalmente significativos.

Sin embargo, la generación exhaustiva de subsecuencias mediante máscaras de bits tiene una complejidad exponencial $O(2^n)$, lo que limita severamente su escalabilidad a medida que aumenta la longitud de las secuencias o el número de estructuras a analizar. Esta limitación motivó el desarrollo de un nuevo enfoque basado en programación dinámica, descrito en el capítulo siguiente.

%% contenido del tercer capítulo


