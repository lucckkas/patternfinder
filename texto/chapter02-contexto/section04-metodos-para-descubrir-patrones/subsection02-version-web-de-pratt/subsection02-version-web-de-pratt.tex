\subsection{Versión Web de Pratt}


La versión web de PRATT está disponible a través del servidor ExPASy, que es parte del Swiss Institute of Bioinformatics. Esta interfaz web permite a los usuarios ejecutar Pratt sin necesidad de instalar software localmente, facilitando su acceso y uso.

Además, el Protein Data Bank (PDB) ofrece recursos para el análisis de estructuras proteicas, y aunque no proporciona directamente una versión web de Pratt, las secuencias y estructuras disponibles en el PDB pueden ser utilizadas como entrada para Pratt. Los usuarios pueden extraer secuencias de proteínas del PDB y utilizarlas en Pratt para identificar patrones conservados.

Si bien Pratt es una herramienta potente para identificar patrones conservados en secuencias proteicas, no está específicamente diseñada para identificar patrones de interacción proteína-ligando. Los patrones descubiertos por Pratt corresponden a secuencias de aminoácidos que están conservadas entre diferentes proteínas, lo que puede incluir sitios de unión a ligandos si estos sitios están conservados.

La identificación de patrones proteína-ligando requiere información sobre las interacciones específicas entre los aminoácidos de la proteína y el ligando. Esto generalmente implica el análisis de datos estructurales 3D, como los disponibles en el PDB, y el uso de herramientas especializadas que consideran la geometría y las propiedades químicas de la interacción.

