\subsection{Pratt}


Pratt  es una herramienta ampliamente utilizada para la identificación de patrones conservados en conjuntos de secuencias proteicas. Mediante la búsqueda de patrones comunes en un conjunto de secuencias.

Pratt utiliza algoritmos que consideran la variabilidad en las secuencias, permitiendo identificar patrones que no necesariamente están perfectamente conservados, pero que presentan similitudes significativas. Esto es especialmente útil en el análisis de familias de proteínas donde la conservación es parcial debido a divergencias evolutivas.

Algunas características clave de Pratt incluyen:

\begin{itemize} 
    \item \textbf{Flexibilidad en la definición de patrones}: Permite especificar parámetros como la longitud mínima y máxima de los patrones, el número mínimo de secuencias que deben contener el patrón, y la posibilidad de incluir posiciones variables o ambiguas. 
    \item \textbf{Incorporación de gaps}: Puede manejar espacios en los patrones, lo que es útil cuando los motivos funcionales están separados por regiones variables. 
     
    \item \textbf{Salida en formato PROSITE}: Los patrones se presentan en la notación PA Line de PROSITE, facilitando su interpretación y uso en otras herramientas bioinformáticas. \end{itemize}

