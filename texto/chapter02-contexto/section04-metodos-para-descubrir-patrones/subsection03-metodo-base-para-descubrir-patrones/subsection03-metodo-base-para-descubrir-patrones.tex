\subsection{Método base para descubrir patrones}
\label{algoritmooriginal}

Este trabajo se inspira en una propuesta secuencial presentada recientemente \cite{francisco}, que emplea un método exhaustivo de generación de subsecuencias para identificar patrones. El algoritmo original opera mediante tres etapas principales:

\textbf{Paso 1: Generación exhaustiva mediante máscaras de bits.}
Para cada secuencia de aminoácidos de longitud $n$, se generan todas las posibles subsecuencias aplicando una máscara de bits. Por ejemplo, para la secuencia \texttt{ABC}, se utilizan las máscaras \texttt{000}, \texttt{001}, \texttt{010}, \texttt{011}, \texttt{100}, \texttt{101}, \texttt{110}, \texttt{111}, generando las subsecuencias \texttt{ABC}, \texttt{ABx}, \texttt{AxC}, \texttt{Axx}, \texttt{xBC}, \texttt{xBx}, \texttt{xxC}, \texttt{xxx}, donde \texttt{x} representa un hueco (gap).

\textbf{Paso 2: Comparación exhaustiva.}
Se comparan todas las subsecuencias generadas entre sí para identificar coincidencias y alineamientos comunes entre las diferentes secuencias de aminoácidos.

\textbf{Paso 3: Formateo y asignación de puntuación.}
Cada patrón identificado se traduce a la notación PA Line, colapsando los huecos consecutivos en expresiones \texttt{x(n)} y manteniendo en mayúsculas los aminoácidos conservados. Finalmente, se calcula una puntuación para cada patrón, dando prioridad a aquellos con mayor cantidad de aminoácidos conservados (mayúsculas), permitiendo ordenar las alternativas.

\textbf{Ejemplo}

Consideremos las cadenas \texttt{ABCD} y \texttt{AHCD}. El método exhaustivo generaría todas las posibles subsecuencias de ambas cadenas y compararía cada par para encontrar coincidencias. Entre los patrones resultantes se encontraría:

\begin{itemize}
    \item \texttt{A-x(1)-C-D}
\end{itemize}

Este patrón captura la coincidencia de \texttt{A}, \texttt{C} y \texttt{D}, con un hueco correspondiente a la diferencia entre \texttt{B} y \texttt{H}.

\vspace{1em}

\textbf{Limitaciones y modificaciones.}
El método original no consideraba los aminoácidos interactuantes con el ligando, procesando secuencias completas sin discriminar qué residuos son biológicamente relevantes para la interacción. Por esta razón, se realizó una primera modificación al algoritmo para incorporar únicamente los aminoácidos que interactúan directamente con cada ligando, reduciendo así el espacio de búsqueda a los segmentos funcionalmente significativos.

Sin embargo, la generación exhaustiva de subsecuencias mediante máscaras de bits tiene una complejidad exponencial $O(2^n)$, lo que limita severamente su escalabilidad a medida que aumenta la longitud de las secuencias o el número de estructuras a analizar. Esta limitación motivó el desarrollo de un nuevo enfoque basado en programación dinámica, descrito en el capítulo siguiente.

%% contenido del tercer capítulo
