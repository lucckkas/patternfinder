\section{Patrones Proteína-Ligando}
\label{sec:Patrones Proteína-Ligando}

Un patrón proteína-ligando es una secuencia específica de aminoácidos alrededor de un ligando que se repite en múltiples proteínas. Estos patrones representan regiones funcionales donde ocurre la interacción entre la proteína y la molécula del ligando, lo cual es esencial para la actividad biológica de la proteína. La identificación y análisis de estos patrones permiten comprender mejor cómo las proteínas llevan a cabo sus funciones.

La detección de estos patrones es fundamental en diversas áreas, como el diseño de fármacos, donde se busca identificar moléculas que puedan unirse eficazmente a una proteína objetivo para modificar su actividad. También es crucial en la ingeniería de proteínas, donde modificar o diseñar nuevas proteínas con funciones específicas requiere un entendimiento detallado de los patrones de interacción proteína-ligando.

Para representar estos patrones en secuencias proteicas, se utilizan herramientas y notaciones especializadas como la PA Line de PROSITE\cite{manualpaline}, que permite describir patrones complejos de manera concisa y estandarizada. Esta notación facilita la comparación y el análisis de secuencias, permitiendo a los investigadores identificar rápidamente regiones de interés y predecir posibles interacciones funcionales.

