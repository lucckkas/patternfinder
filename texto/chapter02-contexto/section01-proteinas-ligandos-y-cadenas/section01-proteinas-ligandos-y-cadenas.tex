\section{Proteínas, ligandos y cadenas}

Las definiciones formales de algunos conceptos importantes son:
\begin{itemize}
    \item Aminoácidos: Son moléculas que se combinan para formar proteínas.\cite{defAminoacido}
    \item Proteínas: Son moléculas grandes y complejas que desempeñan muchas funciones críticas en el cuerpo. Realizan la mayor parte del trabajo en las células y son necesarias para la estructura, función y regulación de los tejidos y órganos del cuerpo.\cite{defProteina}
    \item Ligandos: Es una sustancia (usualmente una molécula pequeña) que forma un complejo con una biomolécula. En un sentido más estricto, es una molécula que envía una señal al unirse al centro activo de una proteína. \cite{defLigando} Es decir, un ligando es una molécula que se une a una proteína, alterando su estructura y función.
\end{itemize}

Hay 20 tipos diferentes de aminoácidos que se pueden combinar para formar una proteína. La secuencia de aminoácidos determina la estructura tridimensional única de cada proteína y su función específica \cite{defProteina}.

El Cuadro \ref{tab:aminos} contiene la lista de aminoácidos estándar.

\begin{table}[H]
\centering
\begin{tabular}{|l|c|c|}
\hline
\textbf{Nombre} & \textbf{Abreviatura} & \textbf{Símbolo} \\
\hline
Alanina & Ala & A \\
Arginina & Arg & R \\
Asparagina & Asn & N \\
Aspártico & Asp & D \\
Cisteína & Cys & C \\
Fenilalanina & Phe & F \\
Glicina & Gly & G \\
Glutámico & Glu & E \\
Glutamina & Gln & Q \\
Histidina & His & H \\
Isoleucina & Ile & I \\
Leucina & Leu & L \\
Lisina & Lys & K \\
Metionina & Met & M \\
Prolina & Pro & P \\
Serina & Ser & S \\
Tirosina & Tyr & Y \\
Treonina & Thr & T \\
Triptófano & Trp & W \\
Valina & Val & V \\
\hline
\end{tabular}
\caption{Lista de aminoácidos estándar.}
\label{tab:aminos}
\end{table}

Un ejemplo de una secuencia de proteína, correspondiente a la proteína 5PNQ \cite{prote5PNQ}, es el siguiente:

\begin{verbatim}
MHHHHHHSSGVDLGTENLYFQSMETMKSKANCAQNPNCNIMIFHPTKEEFNDFDKYIAYMESQG
AHRAGLAKIIPPKEWKARETYDNISEILIATPLQQVASGRAGVFTQYHKKKKAMTVGEYRHLAN
SKKYQTPPHQNFEDLERKYWKNRIYNSPIYGADISGSLFDENTKQWNLGHLGTIQDLLEKECGV
VIEGVNTPYLYFGMWKTTFAWHTEDMDLYSINYLHLGEPKTWYVVPPEHGQRLERLARELFPGS
SRGCGAFLRHKVALISPTVLKENGIPFNRITQEAGEFMVTFPYGYHAGFNHGFNCAEAINFATP
RWIDYGKMASQCSCGEARVTFSMDAFVRILQPERYDLWKRGQDR
\end{verbatim}

Adicionalmente y para facilitar el entendimiento estableceremos el siguiente vocabulario para referirnos a cada parte de la proteína que se obtiene durante el algoritmo:
\begin{itemize}
    \item Secuencia: Para referirnos a una proteína completa.
    \item Subsecuencia de interés: Parte de la proteína que interactúa con un ligando y recortamos.
\end{itemize}

