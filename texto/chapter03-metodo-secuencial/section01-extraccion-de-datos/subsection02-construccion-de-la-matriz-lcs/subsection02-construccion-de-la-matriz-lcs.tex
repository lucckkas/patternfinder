\subsection{Construcción de la matriz LCS}


Para cada par de segmentos que interactúan con un ligando se construye una matriz $(n+1) \times (m+1)$ que almacena la longitud de la \textit{Longest Common Subsequence} entre los prefijos de ambas cadenas. La recurrencia empleada es la usual en programación dinámica:

\[
L_{i,j} = 
\begin{cases}
L_{i-1,j-1} + 1 & \text{si } s_i = t_j, \\
\max(L_{i-1,j}, L_{i,j-1}) & \text{en otro caso}.
\end{cases}
\]

Además de la matriz $L$ se mantiene una matriz de direcciones $D$ que registra si la mejor transición proviene de la diagonal, desde arriba o desde la izquierda. Esta información permite reconstruir posteriormente todos los alineamientos que alcanzan la longitud óptima.

Para los segmentos de ejemplo \texttt{AbC} y \texttt{AgC}, la matriz resultante es:

\[
\begin{array}{c|cccc}
   & - & A & g & C \\
\hline
-  & 0 & 0 & 0 & 0 \\
A  & 0 & 1 & 1 & 1 \\
b  & 0 & 1 & 1 & 1 \\
C  & 0 & 1 & 1 & 2 \\
\end{array}
\]

El valor $L_{3,3} = 2$ indica que la LCS tiene longitud dos y que existen coincidencias suficientes para derivar un patrón candidato útil.

