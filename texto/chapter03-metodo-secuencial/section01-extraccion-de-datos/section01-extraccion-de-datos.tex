\section{Extracción de datos}


Se desarrolló un algoritmo que permite detectar los aminoácidos que interactúan con los ligandos en una proteína y que retorna los segmentos de la proteína que contienen esos aminoácidos.

Para esto se utilizó la librería \texttt{Biopython} \cite{biopython_articulo}, la cual es una colección de Python disponible gratuitamente módulos para biología molecular computacional, y el paquete Bio.PDB \cite{biopdb} que facilita el uso y procesamiento de los archivos \texttt{.pdb}, que corresponden a archivos del Protein Data Bank (PDB) \cite{pdb}.

\subsection{Descripción del Algoritmo}


El algoritmo se compone de varios pasos que se detallan a continuación, junto con ejemplos aplicados a la proteína \texttt{1znf} para facilitar su comprensión.

\subsubsection{Configuración Inicial}


En esta sección, se establecen los parámetros iniciales y se preparan las estructuras de datos necesarias para el análisis.

\begin{algorithm}[H]
\caption{Proceso de análisis de interacción proteína-ligando}
\begin{algorithmic}[1]
\State Establecer el umbral de distancia para considerar una interacción (\texttt{distance\_threshold})
\State Leer el archivo PDB para obtener la estructura de la proteína
\State Inicializar listas para residuos de ligando (\texttt{ligand\_residues}), átomos de proteína (\texttt{protein\_atoms}) y lista de residuos (\texttt{residues\_list})
\end{algorithmic}
\end{algorithm}


\paragraph{Ejemplo:}

Para la proteína \texttt{1znf}, se establece un umbral de distancia de 4.0 \AA. Se lee el archivo \texttt{1znf.pdb} y se inicializan las listas vacías para almacenar los residuos de ligando, átomos de proteína y lista de residuos.


\subsubsection{Recopilación de Información de la Estructura}


Se recorre la secuencia para identificar los ligandos y aminoácidos de la proteína, almacenando la información relevante en listas para su posterior procesamiento.

\begin{algorithm}[H]
\caption{Proceso de extracción de ligandos y átomos de proteína}
\begin{algorithmic}[1]
\ForAll{cadena en la estructura}
    \ForAll{residuo en la cadena}
        \State Obtener \texttt{hetfield}, \texttt{resseq} y \texttt{icode} para identificar el residuo
        // hetfield: Indica el tipo de residuo. 
        
        // resseq: Es el número de secuencia del residuo
        
        // icode: Código de inserción.
        
        \State Obtener el nombre del residuo (\texttt{resname})
        \If{el residuo es un ligando}
            \State Añadir el residuo a \texttt{ligand\_residues}
        \ElsIf{el residuo es de proteína}
            \State Añadir los átomos del residuo a \texttt{protein\_atoms}
            \State Añadir el residuo a \texttt{residues\_list}
        \EndIf
    \EndFor
\EndFor
\end{algorithmic}
\end{algorithm}


\paragraph{Ejemplo:}

Al procesar la proteína \texttt{1znf}, se identifican los ligandos y de aminoácidos. Los ligandos encontrados son \texttt{NH2}, \texttt{ACE} y \texttt{ZN}, mientras que lo demás corresponde a los aminoácidos de la cadena.


\subsubsection{Detección y Procesamiento de Ligandos}


Se verifica si existen ligandos en la estructura y se procesa cada uno de ellos individualmente.

\begin{algorithm}[H]
\caption{Identificación de aminoácidos interactuantes con ligandos}
\begin{algorithmic}[1]
\If{no se encontraron residuos de ligando}
    \State Terminar el algoritmo
\EndIf
\State Obtener códigos únicos de ligandos (\texttt{ligand\_codes})
\ForAll{código de ligando \texttt{lig\_code} en \texttt{ligand\_codes}}
    \State Obtener los átomos del ligando actual
    \State Proceder con la identificación de aminoácidos interactuantes
\EndFor
\end{algorithmic}
\end{algorithm}


\paragraph{Ejemplo:}

En la proteína \texttt{1znf}, se encontraron los ligandos \texttt{NH2}, \texttt{ACE} y \texttt{ZN}. El algoritmo procesará cada uno de estos ligandos de forma individual.


\subsubsection{Identificación de Aminoácidos Interactuantes}


Utilizando un algoritmo de búsqueda de vecinos (NeighborSearch), se identifican los aminoácidos de la proteína que interactúan con el ligando, basándose en el umbral de distancia definido. Esto consiste en comparar la distancia de cada átomo del ligando con los átomos de la proteína y determinar si algún átomo del aminoácido está a una distancia menor al umbral. Un residuo se considera interactuante si al menos uno de sus átomos está dentro del umbral de distancia respecto a cualquier átomo del ligando.

\begin{verbatim}
Crear una búsqueda de vecinos (NeighborSearch) con los átomos de proteína
Inicializar un conjunto de residuos que interactúan (interacting_residues)

Para cada átomo en los átomos del ligando:
    Buscar residuos vecinos dentro del umbral de distancia
    Añadir estos residuos a interacting_residues
\end{verbatim}

\paragraph{Ejemplo: Identificación de Aminoácidos Interactuantes}

Al procesar el ligando \texttt{NH2} de la proteína \texttt{1znf}, el algoritmo identifica los aminoácidos que se encuentran dentro de 4.0 \AA\ de cualquier átomo del ligando. Los aminoácidos identificados corresponden a los aminoácidos \texttt{H}, \texttt{K} y \texttt{N} al final de la secuencia.


\subsubsection{Generación de la Secuencia Modificada}


Se genera la secuencia completa de aminoácidos de la proteína, utilizando letras mayúsculas para los aminoácidos que interactúan con el ligando y minúsculas para los que no.

\begin{verbatim}
Inicializar la secuencia resaltada (highlighted_sequence)

Para cada residuo en residues_list:
    Obtener el nombre del residuo (resname)
    Convertir resname a código de una letra (one_letter)
    
    Si el residuo está en interacting_residues:
        Convertir one_letter a mayúscula
    Sino:
        Convertir one_letter a minúscula
    
    Añadir one_letter a highlighted_sequence
\end{verbatim}


\paragraph{Ejemplo: Generación de la Secuencia Modificada}

Con la secuencia original de la proteína \texttt{1znf}:

\begin{center}
\texttt{ykcglcersfveksalsrhqrvhkn}
\end{center}

Y considerando que los residuos que interactúan con el ligando \texttt{NH2} son \texttt{H}, \texttt{K} y \texttt{N}, la secuencia modificada será:

\begin{center}
\texttt{ykcglcersfveksalsrhqrvHKN}
\end{center}

Las letras mayúsculas indican los aminoácidos que interactúan con el ligando.


\subsubsection{Extracción y Presentación del Segmento Interactuante}


Se extrae el segmento de la secuencia que abarca desde el primer hasta el último aminoácido que interactúa con el ligando.

\begin{verbatim}
Utilizar una expresión regular para encontrar el segmento que inicia
y termina con una letra mayúscula

Si se encuentra una coincidencia:
    Extraer el segmento interactuante (interacting_segment)
    Mostrar el segmento interactuante al usuario
Sino:
    Indicar que no se encontraron residuos que interactúen con el ligando
\end{verbatim}


\paragraph{Ejemplo: Extracción y Presentación del Segmento Interactuante}

Aplicando la expresión regular a la secuencia modificada:

\begin{center}
\texttt{ykcglcersfveksalsrhqrvHKN}
\end{center}

Se extrae el segmento que va desde la primera letra mayúscula hasta la última:

\begin{center}
\texttt{HKN}
\end{center}

Este es el segmento de aminoácidos que interactúa con el ligando \texttt{NH2}.
El mismo proceso se repite para los otros ligandos identificados. En caso de que los aminoácidos no sean consecutivos, la subsecuencia de interés incluira los aminoácidos intermedios que no interactúan con el ligando en minúsculas. Por ejemplo, si la secuencia fuera:
\begin{center}
    \texttt{ykcglcersHfveksalKsrhqrvN}
\end{center}
El segmento interactuante extraído sería:
\begin{center}
    \texttt{HfveksalKsrhqrvN}
\end{center}

\subsection{Construcción de la matriz LCS}


Para cada par de segmentos que interactúan con un ligando se construye una matriz $(n+1) \times (m+1)$ que almacena la longitud de la \textit{Longest Common Subsequence} entre los prefijos de ambas cadenas. La recurrencia empleada es la usual en programación dinámica:

\[
L_{i,j} = 
\begin{cases}
L_{i-1,j-1} + 1 & \text{si } s_i = t_j, \\
\max(L_{i-1,j}, L_{i,j-1}) & \text{en otro caso}.
\end{cases}
\]

Además de la matriz $L$ se mantiene una matriz de direcciones $D$ que registra si la mejor transición proviene de la diagonal, desde arriba o desde la izquierda. Esta información permite reconstruir posteriormente todos los alineamientos que alcanzan la longitud óptima.

Para los segmentos de ejemplo \texttt{AbC} y \texttt{AgC}, la matriz resultante es:

\[
\begin{array}{c|cccc}
   & - & A & g & C \\
\hline
-  & 0 & 0 & 0 & 0 \\
A  & 0 & 1 & 1 & 1 \\
b  & 0 & 1 & 1 & 1 \\
C  & 0 & 1 & 1 & 2 \\
\end{array}
\]

El valor $L_{3,3} = 2$ indica que la LCS tiene longitud dos y que existen coincidencias suficientes para derivar un patrón candidato útil.


\subsection{Recuperación de patrones candidatos}


La reconstrucción de los patrones se realiza mediante un recorrido inverso sobre $D$. Cada vez que existen múltiples direcciones posibles se exploran todas las alternativas para obtener alineamientos diferentes. El proceso genera secuencias intermedias en las que se alternan coincidencias exactas con huecos que representan los aminoácidos no compartidos entre ambas cadenas.

En el ejemplo anterior, el retroceso toma una diagonal al encontrar la coincidencia \texttt{C}, avanza hacia arriba para contabilizar un hueco de longitud uno y finaliza con la coincidencia \texttt{A}. El alineamiento resultante se escribe:
\begin{itemize}
    \item \texttt{A-C}
\end{itemize}


\subsection{Formateo de patrones}


Cada alineamiento se transforma a la notación PA Line de Prosite. Los aminoácidos conservados se expresan con letras mayúsculas y los huecos se convierten en expresiones \texttt{x(n)}, donde $n$ es la longitud del segmento ausente. Utilizando el alineamiento previo se obtiene:

\begin{itemize}
    \item \texttt{A-g-C} $\rightarrow$ \texttt{A-x(1)-C}
    \item \texttt{A} $\rightarrow$ \texttt{A}
    \item \texttt{C} $\rightarrow$ \texttt{C}
\end{itemize}

Además, se define una notación complementaria para los casos en que un patrón presenta segmentos ausentes de longitudes $X_i$ observadas pero no de una longitud $X_j$ determinada. En tales situaciones se registran explícitamente las longitudes detectadas; por ejemplo, la notación

\texttt{A-x(1|2|3|5)-C}

indica que el patrón admite un hueco de longitud 1, 2, 3 o 5, y que no se ha observado un hueco de longitud 4.


\subsection{Asignación de Puntuaciones}


A cada patrón se le asigna una puntuación basada en la siguiente fórmula:

\[
\text{Puntuación} = x + m \times n + y \times n^2
\]

Donde:

\begin{itemize}
    \item $x$: Número de \texttt{x} (gaps) en el patrón.
    \item $m$: Número de letras minúsculas en el patrón.
    \item $y$: Número de letras mayúsculas en el patrón.
    \item $n$: Longitud del segmento original.
\end{itemize}

Para nuestro ejemplo, con segmentos de longitud $n=3$, calculamos las puntuaciones:

\begin{itemize}
    \item \texttt{A-x(1)-C}:
    \begin{itemize}
        \item $x = 1$ (un gap)
        \item $m = 0$ (cero minúsculas)
        \item $y = 2$ (A y C son mayúsculas)
        \item Puntuación: $1 + 0 \times 3 + 2 \times 3^2 = 1 + 0 + 18 = 19$
    \end{itemize}
    \item \texttt{A}:
    \begin{itemize}
        \item $x = 0$
        \item $m = 0$
        \item $y = 1$
        \item Puntuación: $0 + 0 \times 3 + 1 \times 3^2 = 0 + 0 + 9 = 9$
    \end{itemize}
    \item \texttt{C}:
    \begin{itemize}
        \item $x = 0$
        \item $m = 0$
        \item $y = 1$
        \item Puntuación: $0 + 0 \times 3 + 1 \times 3^2 = 0 + 0 + 9 = 9$
    \end{itemize}
\end{itemize}


\subsection{Ordenamiento de Patrones}


Finalmente, se ordenan los patrones según su puntuación de mayor a menor:

\begin{enumerate}
    \item \texttt{A-x(1)-C} ($19$ puntos)
    \item \texttt{A} ($9$ puntos)
    \item \texttt{C} ($9$ puntos)
\end{enumerate}

Siendo el patrón A-x(1)-C el elegido como mejor patrón.



