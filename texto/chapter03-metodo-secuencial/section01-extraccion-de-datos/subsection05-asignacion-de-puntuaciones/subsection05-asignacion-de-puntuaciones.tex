\subsection{Asignación de Puntuaciones}


A cada patrón se le asigna una puntuación basada en la siguiente fórmula:

\[
\text{Puntuación} = x + m \times n + y \times n^2
\]

Donde:

\begin{itemize}
    \item $x$: Número de \texttt{x} (gaps) en el patrón.
    \item $m$: Número de letras minúsculas en el patrón.
    \item $y$: Número de letras mayúsculas en el patrón.
    \item $n$: Longitud del segmento original.
\end{itemize}

Para nuestro ejemplo, con segmentos de longitud $n=3$, calculamos las puntuaciones:

\begin{itemize}
    \item \texttt{A-x(1)-C}:
    \begin{itemize}
        \item $x = 1$ (un gap)
        \item $m = 0$ (cero minúsculas)
        \item $y = 2$ (A y C son mayúsculas)
        \item Puntuación: $1 + 0 \times 3 + 2 \times 3^2 = 1 + 0 + 18 = 19$
    \end{itemize}
    \item \texttt{A}:
    \begin{itemize}
        \item $x = 0$
        \item $m = 0$
        \item $y = 1$
        \item Puntuación: $0 + 0 \times 3 + 1 \times 3^2 = 0 + 0 + 9 = 9$
    \end{itemize}
    \item \texttt{C}:
    \begin{itemize}
        \item $x = 0$
        \item $m = 0$
        \item $y = 1$
        \item Puntuación: $0 + 0 \times 3 + 1 \times 3^2 = 0 + 0 + 9 = 9$
    \end{itemize}
\end{itemize}

