\subsubsection{Generación de la Secuencia Modificada}


Se genera la secuencia completa de aminoácidos de la proteína, utilizando letras mayúsculas para los aminoácidos que interactúan con el ligando y minúsculas para los que no.

\begin{algorithm}[H]
\caption{Generación de secuencia resaltada de residuos}
\begin{algorithmic}[1]
\State Inicializar la secuencia resaltada (\texttt{highlighted\_sequence})
\ForAll{residuo en \texttt{residues\_list}}
    \State Obtener el nombre del residuo (\texttt{resname})
    \State Convertir \texttt{resname} a código de una letra (\texttt{one\_letter})
    \If{el residuo está en \texttt{interacting\_residues}}
        \State Convertir \texttt{one\_letter} a mayúscula
    \Else
        \State Convertir \texttt{one\_letter} a minúscula
    \EndIf
    \State Añadir \texttt{one\_letter} a \texttt{highlighted\_sequence}
\EndFor
\end{algorithmic}
\end{algorithm}


\paragraph{Ejemplo: Generación de la Secuencia Modificada}

Con la secuencia original de la proteína \texttt{1znf}:

\begin{center}
\texttt{ykcglcersfveksalsrhqrvhkn}
\end{center}

Y considerando que los residuos que interactúan con el ligando \texttt{NH2} son \texttt{H}, \texttt{K} y \texttt{N}, la secuencia modificada será:

\begin{center}
\texttt{ykcglcersfveksalsrhqrvHKN}
\end{center}

Las letras mayúsculas indican los aminoácidos que interactúan con el ligando.

