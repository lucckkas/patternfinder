\subsubsection{Generación de la Secuencia Modificada}


Se genera la secuencia completa de aminoácidos de la proteína, utilizando letras mayúsculas para los aminoácidos que interactúan con el ligando y minúsculas para los que no.

\begin{verbatim}
Inicializar la secuencia resaltada (highlighted_sequence)

Para cada residuo en residues_list:
    Obtener el nombre del residuo (resname)
    Convertir resname a código de una letra (one_letter)
    
    Si el residuo está en interacting_residues:
        Convertir one_letter a mayúscula
    Sino:
        Convertir one_letter a minúscula
    
    Añadir one_letter a highlighted_sequence
\end{verbatim}


\paragraph{Ejemplo: Generación de la Secuencia Modificada}

Con la secuencia original de la proteína \texttt{1znf}:

\begin{center}
\texttt{ykcglcersfveksalsrhqrvhkn}
\end{center}

Y considerando que los residuos que interactúan con el ligando \texttt{NH2} son \texttt{H}, \texttt{K} y \texttt{N}, la secuencia modificada será:

\begin{center}
\texttt{ykcglcersfveksalsrhqrvHKN}
\end{center}

Las letras mayúsculas indican los aminoácidos que interactúan con el ligando.

