\subsubsection{Extracción y Presentación del Segmento Interactuante}


Se extrae el segmento de la secuencia que abarca desde el primer hasta el último aminoácido que interactúa con el ligando.

\begin{algorithm}[H]
\caption{Detección de segmento interactuante en la secuencia resaltada}
\begin{algorithmic}[1]
\State Utilizar una expresión regular para encontrar el segmento que inicia y termina con una letra mayúscula
\If{se encuentra una coincidencia}
    \State Extraer el segmento interactuante (\texttt{interacting\_segment})
    \State Mostrar el segmento interactuante al usuario
\Else
    \State Indicar que no se encontraron residuos que interactúen con el ligando
\EndIf
\end{algorithmic}
\end{algorithm}
 ligando y se muestra al usuario.

\begin{algorithm}[H]
\caption{Detección de segmento interactuante en la secuencia resaltada}
\begin{algorithmic}[1]
\State Utilizar una expresión regular para encontrar el segmento que inicia y termina con una letra mayúscula
\If{se encuentra una coincidencia}
    \State Extraer el segmento interactuante (\texttt{interacting\_segment})
    \State Mostrar el segmento interactuante al usuario
\Else
    \State Indicar que no se encontraron residuos que interactúen con el ligando
\EndIf
\end{algorithmic}
\end{algorithm}


\paragraph{Ejemplo: Extracción y Presentación del Segmento Interactuante}

Aplicando la expresión regular a la secuencia modificada:

\begin{center}
\texttt{ykcglcersfveksalsrhqrvHKN}
\end{center}

Se extrae el segmento que va desde la primera letra mayúscula hasta la última:

\begin{center}
\texttt{HKN}
\end{center}

Este es el segmento de aminoácidos que interactúa con el ligando \texttt{NH2}.
El mismo proceso se repite para los otros ligandos identificados.

