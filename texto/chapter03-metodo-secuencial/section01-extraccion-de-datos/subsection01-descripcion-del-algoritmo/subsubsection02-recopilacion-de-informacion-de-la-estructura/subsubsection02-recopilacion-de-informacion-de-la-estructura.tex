\subsubsection{Recopilación de Información de la Estructura}


Se recorre la secuencia para identificar los ligandos y aminoácidos de la proteína, almacenando la información relevante en listas para su posterior procesamiento.

\begin{algorithm}[H]
\caption{Proceso de extracción de ligandos y átomos de proteína}
\begin{algorithmic}[1]
\ForAll{cadena en la estructura}
    \ForAll{residuo en la cadena}
        \State Obtener \texttt{hetfield}, \texttt{resseq} y \texttt{icode} para identificar el residuo
        // hetfield: Indica el tipo de residuo. 
        
        // resseq: Es el número de secuencia del residuo
        
        // icode: Código de inserción.
        
        \State Obtener el nombre del residuo (\texttt{resname})
        \If{el residuo es un ligando}
            \State Añadir el residuo a \texttt{ligand\_residues}
        \ElsIf{el residuo es de proteína}
            \State Añadir los átomos del residuo a \texttt{protein\_atoms}
            \State Añadir el residuo a \texttt{residues\_list}
        \EndIf
    \EndFor
\EndFor
\end{algorithmic}
\end{algorithm}


\paragraph{Ejemplo:}

Al procesar la proteína \texttt{1znf}, se identifican los ligandos y de aminoácidos. Los ligandos encontrados son \texttt{NH2}, \texttt{ACE} y \texttt{ZN}, mientras que lo demás corresponde a los aminoácidos de la cadena.

