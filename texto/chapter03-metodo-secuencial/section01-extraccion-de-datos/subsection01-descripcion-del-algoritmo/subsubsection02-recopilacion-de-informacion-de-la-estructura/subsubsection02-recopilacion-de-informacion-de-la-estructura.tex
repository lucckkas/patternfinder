\subsubsection{Recopilación de Información de la Estructura}


Se recorre la secuencia para identificar los ligandos y aminoácidos de la proteína, almacenando la información relevante en listas para su posterior procesamiento.

\begin{verbatim}
Para cada cadena en la estructura:
Para cada residuo en la cadena:
Obtener hetfield, resseq e icode para identificar el residuo
// hetfield: Indica el tipo de residuo
// resseq: Número de secuencia del residuo
// icode: Código de inserción
        
        Obtener el nombre del residuo (resname)
        
        Si el residuo es un ligando:
        Añadir el residuo a ligand_residues
        Sino, si el residuo es de proteína:
        Añadir los átomos del residuo a protein_atoms
        Añadir el residuo a residues_list
        \end{verbatim}


\paragraph{Ejemplo:}

Al procesar la proteína \texttt{1znf}, se identifican los ligandos y de aminoácidos. Los ligandos encontrados son \texttt{NH2}, \texttt{ACE} y \texttt{ZN}, mientras que lo demás corresponde a los aminoácidos de la cadena.

