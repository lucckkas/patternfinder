\maketitle

\begin{dedicatory}
A mis padres y hermano,\\
por su apoyo incondicional\\
en cada paso de este camino.
\end{dedicatory}

\begin{acknowledgment}
Quisiera expresar mi más sincero agradecimiento a todas las personas que hicieron posible la realización de este trabajo.

A mis padres, por su amor, paciencia y apoyo constante durante todos estos años de estudio. Por creer en mí incluso en los momentos más difíciles y por brindarme las herramientas necesarias para alcanzar mis metas.

A mi hermano, por su compañía, sus palabras de aliento y por recordarme la importancia de mantener el equilibrio entre el trabajo y la vida personal.

A mis amigos, por su comprensión en los momentos de ausencia, por las conversaciones que aligeraron la carga del trabajo y por celebrar conmigo cada pequeño logro en este proceso.

A mi profesor guía, por su orientación, dedicación y valiosos aportes que enriquecieron significativamente este trabajo. Su experiencia y conocimientos fueron fundamentales para el desarrollo de esta investigación.

A mis compañeros de carrera, por compartir este viaje académico, por las largas horas de estudio conjunto y por crear un ambiente de apoyo mutuo que hizo más llevadero este desafío.

A todos ustedes, gracias por ser parte de este importante logro en mi vida.
\end{acknowledgment}

\tableofcontents
\listoffigures
\listoftables

\begin{resumen}
La identificación de patrones en proteínas es un desafío clave en bioinformática, esencial para comprender cómo interactúan los aminoácidos con otras moléculas como los ligandos. Las aproximaciones tradicionales sufren un crecimiento combinatorio que limita su uso práctico debido al elevado coste computacional.

Para abordar esta problemática se desarrolló un algoritmo paralelo basado en la estrategia de la \textit{Longest Common Subsequence} (LCS). La solución reemplaza la enumeración exhaustiva de subsecuencias por una formulación de programación dinámica que calcula las coincidencias sobre una matriz $n \times m$ y permite recuperar alineamientos compatibles con la notación Prosite. La implementación, construida en Go y apoyada en Biopython para el preprocesamiento, distribuye el cálculo y la reconstrucción de la LCS entre múltiples hilos para reducir los tiempos de búsqueda frente al enfoque secuencial.

Si bien el algoritmo LCS-paralelo mitiga la explosión combinatoria del método previo, su eficiencia sigue dependiendo de la longitud de las secuencias, del número de alineamientos necesarios y de las características del hardware disponible. Futuras optimizaciones pueden considerar técnicas avanzadas de poda y heurísticas de alineamiento para escalar a volúmenes aún mayores sin perder precisión biológica.

La proyección de este trabajo es promisoria en contextos como el diseño de fármacos y la ingeniería de proteínas, donde se requiere identificar patrones proteína-ligando de manera rápida y confiable. También se contempla la integración con otras herramientas bioinformáticas para habilitar análisis complejos en tiempo real y extender la aplicabilidad del algoritmo a estudios de mayor envergadura.
\end{resumen}
