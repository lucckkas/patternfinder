\section{Paralelización de la construcción de la matriz}

El algorimo paraleliza el llenado de la matriz procesando las diagonales (de arriba-derecha a abajo-izquierda) de forma secuencial, pero calculando todas las celdas de cada diagonal en paralelo. Las celdas en una misma diagonal tienen la propiedad de que sus índices $(i, j)$ suman un valor constante. La estrategia es la siguiente:

\begin{verbatim}
Para cada diagonal de la matriz:
    Para cada celda (i,j) en la diagonal:
        Crear una tarea paralela que:
            Si sec1[i-1] == sec2[j-1]:
                dp[i][j] = dp[i-1][j-1] + 1
            Sino:
                dp[i][j] = max(dp[i-1][j], dp[i][j-1])
    
    Esperar a que todas las tareas terminen
    Continuar con la siguiente diagonal
\end{verbatim}

La clave está en que cada celda solo depende de valores de la diagonal anterior (ya calculados), permitiendo procesar toda una diagonal de forma concurrente. El mecanismo de sincronización (\texttt{WaitGroup}) garantiza que una diagonal se complete antes de iniciar la siguiente, respetando las dependencias de la programación dinámica.
