\section{Motivación y problemática}

La creciente disponibilidad de datos biológicos y la necesidad de análisis rápidos en bioinformática subrayan la importancia de desarrollar algoritmos eficientes. La capacidad de identificar patrones proteína-ligando de manera rápida y precisa es crucial para avances en áreas como el diseño de fármacos y la comprensión de interacciones moleculares. Este proyecto busca abordar estas necesidades mediante la implementación de un algoritmo paralelo basado en LCS, optimizando tanto el rendimiento como la escalabilidad del proceso de búsqueda de patrones.

Un trabajo previo a este proyecto \cite{francisco} recurría a la generación exhaustiva de subsecuencias y a su comparación posterior, lo que producía un crecimiento combinatorio que limitaba la escalabilidad del método. La necesidad de un algoritmo que mantenga la calidad de los patrones pero reduzca drásticamente el número de comparaciones motivó la adopción del enfoque LCS. Replantear el problema en términos de programación dinámica permite acotar la complejidad a $O(nm)$, mejorar el aprovechamiento de la computación paralela y habilitar análisis oportunos en estudios con grandes volúmenes de datos.