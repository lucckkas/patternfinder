\section{Motivación y problemática}

La versión inicial de este proyecto recurría a la generación exhaustiva de subsecuencias y a su comparación posterior, lo que producía un crecimiento combinatorio que limitaba la escalabilidad del método. La necesidad de un algoritmo que mantenga la calidad de los patrones pero reduzca drásticamente el número de comparaciones motivó la adopción del enfoque LCS. Replantear el problema en términos de programación dinámica permite acotar la complejidad a $O(nm)$, mejorar el aprovechamiento de la computación paralela y habilitar análisis oportunos en estudios con grandes volúmenes de datos.

