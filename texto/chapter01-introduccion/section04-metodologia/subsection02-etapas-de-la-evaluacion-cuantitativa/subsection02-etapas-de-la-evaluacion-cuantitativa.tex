\subsection{Etapas de la Evaluación Cuantitativa}

La evaluación cuantitativa complementa el uso del método científico mediante la recolección y análisis de datos numéricos para medir y validar los resultados:

\paragraph{Definición de métricas} Se identificarán las métricas clave, como el tiempo de ejecución y el uso de recursos.

\paragraph{Recolección de datos} Durante la etapa de experimentación, se registrarán los tiempos de ejecución y el uso de recursos del sistema en múltiples escenarios de prueba.

\paragraph{Análisis estadístico} Los datos recolectados serán analizados estadísticamente para identificar patrones, diferencias significativas entre la versión secuencial y paralelizada, y la eficiencia de la paralelización. Este análisis permitirá validar si la paralelización logra mejorar el rendimiento del algoritmo conforme a los objetivos del proyecto.

\paragraph{Interpretación y presentación} Los resultados del análisis cuantitativo se interpretarán y se incluirán en la documentación final del proyecto, mostrando de forma clara el impacto de la paralelización en el rendimiento del algoritmo.

Este enfoque metodológico, que combina el Método Científico con una evaluación cuantitativa detallada, asegura que el proyecto se desarrolle de manera sistemática y objetiva, permitiendo validar de forma rigurosa las mejoras obtenidas y la consecución de los objetivos propuestos.

