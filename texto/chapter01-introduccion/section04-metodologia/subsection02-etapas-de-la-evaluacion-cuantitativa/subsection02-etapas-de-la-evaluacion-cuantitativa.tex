\subsection{Etapas de la Evaluación Cuantitativa}

La evaluación cuantitativa complementó el uso del método científico mediante la recolección y análisis de datos numéricos para medir y validar los resultados:

\paragraph{Definición de métricas} Se identificaron las métricas clave, como el tiempo de ejecución total, tiempo de construcción de la matriz DP, tiempo de backtracking, speedup, eficiencia, y número de LCS encontradas.

\paragraph{Recolección de datos} Durante la etapa de experimentación, se registraron los tiempos de ejecución y el uso de recursos del sistema en múltiples escenarios de prueba, evaluando secuencias de longitud 20 hasta 260 caracteres.

\paragraph{Análisis estadístico} Los datos recolectados fueron analizados estadísticamente para identificar patrones, diferencias significativas entre la versión secuencial y paralelizada, y la eficiencia de la paralelización. Este análisis permitió validar que la paralelización interna del algoritmo LCS presentó overhead significativo, mientras que la paralelización a nivel de comparaciones pareadas logró speedups de hasta 3.59x.

\paragraph{Interpretación y presentación} Los resultados del análisis cuantitativo se interpretaron y se incluyeron en la documentación final del proyecto, mostrando de forma clara el impacto diferenciado de ambas estrategias de paralelización y proporcionando recomendaciones prácticas para futuros desarrollos.

Este enfoque metodológico, que combinó el Método Científico con una evaluación cuantitativa detallada, aseguró que el proyecto se desarrollara de manera sistemática y objetiva, permitiendo validar de forma rigurosa tanto los resultados esperados como los inesperados, contribuyendo al conocimiento sobre cuándo y cómo aplicar paralelización efectivamente.

