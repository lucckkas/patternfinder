\subsection{Etapas del Método Científico}

El proyecto se desarrollará siguiendo las etapas clásicas del método científico, asegurando un enfoque sistemático y objetivo:

\paragraph{Observación} Se inicia con la revisión y análisis del algoritmo secuencial existente, identificando el problema central: la explosión combinatoria que limita su eficiencia. Esta etapa implica recopilar datos y entender a fondo el comportamiento actual del algoritmo.

\paragraph{Formulación de hipótesis} Basándose en la observación, se planteará la hipótesis de que la paralelización mediante hilos mejorará significativamente el tiempo de ejecución del algoritmo sin comprometer la calidad de los resultados.

\paragraph{Experimentación} Se implementará la versión paralelizada del algoritmo y se llevarán a cabo experimentos para poner a prueba la hipótesis. Esta etapa incluirá la ejecución del algoritmo bajo diferentes configuraciones, variando el número de hilos y analizando cómo se distribuye la carga de trabajo.

\paragraph{Análisis de resultados} Los datos obtenidos durante la experimentación se analizarán para verificar si la paralelización cumple con las expectativas de mejora de rendimiento. Se compararán los tiempos de ejecución y el uso de recursos entre la versión secuencial y la paralela.

\paragraph{Conclusiones} Basándose en el análisis de los resultados, se extraerán conclusiones sobre la efectividad de la paralelización. Si los resultados son positivos, se documentará el éxito del enfoque; en caso contrario, se propondrán nuevas iteraciones o ajustes.

