\section{Contexto}

Los algoritmos para comparar secuencias de aminoácidos suelen apoyarse en técnicas de programación dinámica como la \textit{Longest Common Subsequence} (LCS) para identificar alineamientos significativos. Este enfoque permite capturar coincidencias relevantes sin necesidad de explorar cada subsecuencia de forma explícita, lo que lo convierte en una herramienta ampliamente utilizada en bioinformática para analizar similitudes estructurales y funcionales entre proteínas.

