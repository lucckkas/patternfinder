\section{Contexto}
Las comparaciones entre secuencias biológicas enfrentan un crecimiento combinatorio rápido a medida que aumenta la longitud de las secuencias y el número de secuencias a comparar. El espacio de posibles alineamientos (incluyendo distintas posiciones de coincidencias, sustituciones y huecos) se expande de forma exponencial, por lo que las búsquedas exhaustivas dejan de ser prácticas para longitudes moderadas. Esto plantea retos computacionales importantes: incluso cuando los métodos exactos son factibles para pares cortos, la complejidad y el coste en tiempo y memoria crecen hasta niveles inabordables para problemas más grandes. En la práctica se recurre a estrategias que reducen el espacio de búsqueda (heurísticas, algoritmos aproximados, índices, filtrado previo y modelos probabilísticos) que negocian entre precisión y eficiencia para obtener resultados útiles en tiempos razonables.

