\section{Contribuciones}

La propuesta aporta tres contribuciones principales. Primero, redefine el problema de búsqueda de patrones proteína-ligando en términos de programación dinámica, lo que permite mantener la correctitud de los patrones y, al mismo tiempo, acotar el coste computacional frente al método previo basado en enumeraciones. Segundo, incorpora un flujo automatizado que va desde la extracción de segmentos interactuantes hasta la generación de patrones anotados con los valores de gaps observados en cada cadena, facilitando su interpretación en formato Prosite PA Line \cite{manualpaline}. Finalmente, habilita una base sobre la cual se despliega una version paralela y perfil de rendimiento que cuantifica qué etapas requieren mayor optimización. Estas contribuciones habilitan análisis más rápidos sin sacrificar trazabilidad ni rigor biológico, y sirven de punto de partida para experimentos posteriores descritos en los capítulos siguientes.
