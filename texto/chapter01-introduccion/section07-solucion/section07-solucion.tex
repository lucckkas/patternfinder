\section{Solución}

La solución propuesta implementa un flujo completo para identificar patrones proteína-ligando partiendo de datos estructurales. En primer lugar, se extraen los segmentos de los residuos que interactúan con cada ligando y se filtran las letras minúsculas para conservar únicamente los aminoácidos relevantes. Luego, las parejas de segmentos se comparan mediante la formulación de \textit{Longest Common Subsequence} (LCS), lo que evita enumerar todas las subsecuencias posibles y reduce drásticamente la complejidad combinatoria. A partir de la matriz LCS se reconstruyen todas las coincidencias óptimas y se traducen a la notación Prosite, calculando además los valores de gaps compatibles en ambas cadenas. Esta descripción introduce los pasos generales; en los capítulos siguientes se desarrolla cada uno con mayor detalle.
