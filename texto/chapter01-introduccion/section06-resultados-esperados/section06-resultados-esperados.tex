\section{Resultados Esperados}

\paragraph{Objetivos específicos}

\subparagraph{Objetivo Específico 1: Identificar las partes del algoritmo que pueden ser paralelizadas.}
\begin{itemize}
    \item \textbf{Resultado Esperado:} Se espera identificar de manera clara las secciones del algoritmo secuencial que presentan alta carga computacional y que pueden beneficiarse de la paralelización.
    \item \textbf{Pruebas para Lograr Resultado Esperado:} Identificación de cuellos de botella a través de pruebas de rendimiento.
\end{itemize}

\subparagraph{Objetivo Específico 2: Diseñar un algoritmo que haga uso de hilos para descubrir patrones usando computación paralela.}
\begin{itemize}
    \item \textbf{Resultado Esperado:} Se espera tener un diseño detallado y optimizado del algoritmo que divida el trabajo en tareas paralelas de forma efectiva, manteniendo la precisión del análisis.
    \item \textbf{Pruebas para Lograr Resultado Esperado:} Validación del diseño mediante revisiones por pares y simulaciones de ejecución en entornos controlados para evaluar la distribución de carga y sincronización de los hilos.
\end{itemize}

\subparagraph{Objetivo Específico 3: Implementar el algoritmo basado en hilos.}
\begin{itemize}
    \item \textbf{Resultado Esperado:} Se espera que el algoritmo sea implementado con éxito y que ejecute de forma paralela el cálculo de las matrices LCS y la reconstrucción de patrones sin errores de concurrencia.
    \item \textbf{Pruebas para Lograr Resultado Esperado:} Ejecuciones de prueba del código en entornos de desarrollo, uso de pruebas unitarias para validar la funcionalidad de cada parte del código, y monitoreo del uso de CPU y memoria durante la ejecución.
\end{itemize}

\subparagraph{Objetivo Específico 4: Evaluar el rendimiento del algoritmo en paralelo frente a la versión en serie.}
\begin{itemize}
    \item \textbf{Resultado Esperado:} Se espera que el algoritmo en paralelo reduzca significativamente el tiempo de procesamiento en al menos un 50\% respecto a la versión en serie, sin comprometer la precisión de los resultados.
    \item \textbf{Pruebas para Lograr Resultado Esperado:} Comparación de tiempos de ejecución entre la versión paralela y la versión secuencial del algoritmo en diferentes conjuntos de datos de prueba, y análisis de la eficiencia de los hilos en función del número de hilos usados.
\end{itemize}


