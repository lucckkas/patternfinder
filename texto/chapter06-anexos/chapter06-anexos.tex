\chapter{Anexos}



%% ambiente glosario
\begin{glosario}
  \item[El primer término:] Este es el significado del primer término, realmente no se bien lo que significa pero podría haberlo averiguado si hubiese tenido un poco mas de tiempo.
  \item[El segundo término:] Este si se lo que significa pero me da lata escribirlo...
\end{glosario}

%% genera las referencias (placeholders temporales)
\begin{thebibliography}{99}
\bibitem{biopython_articulo}
  \textbf{[Pendiente]} Referencia real para Biopython. Reemplazar este placeholder por la cita correcta.
\bibitem{biopdb}
  \textbf{[Pendiente]} Referencia real para Bio.PDB. Sustituir por la fuente correspondiente.
\bibitem{pdb}
  \textbf{[Pendiente]} Fuente oficial del Protein Data Bank (PDB). Actualizar con la referencia definitiva.
\bibitem{defLigando}
  \textbf{[Pendiente]} Definición formal de aminoácidos. Cambiar por la cita verificada.
\bibitem{defProteina}
  \textbf{[Pendiente]} Definición formal de proteínas. Cambiar por la cita verificada.
\bibitem{3}
  \textbf{[Pendiente]} Definición detallada de ligandos. Reemplazar por referencia final.
\bibitem{prote5PNQ}
  \textbf{[Pendiente]} Información sobre la proteína 5PNQ. Sustituir por la fuente correcta.
\bibitem{manualpaline}
  \textbf{[Pendiente]} Manual o documentación de la notación PA Line de PROSITE. Reemplazar con la referencia adecuada.
\bibitem{francisco}
  \textbf{[Pendiente]} Trabajo previo de Francisco que sirve de base. Actualizar con la cita oficial.
\end{thebibliography}


%% comienzo de la parte de anexos
\appendixpart

%% contenido del primer anexo
\appendix{El Primer Anexo}
Aquí va el texto del primer anexo...

\section{La primera sección del primer anexo}

Aquí va el texto de la primera sección del primer anexo...


\section{La segunda sección del primer anexo}

Aquí va el texto de la segunda sección del primer anexo...

\subsection{La primera subsección de la segunda sección del primer anexo}



%% contenido del segundo anexo
\appendix{El segundo Anexo}
Aquí va el texto del segundo anexo...



\section{La primera sección del segundo anexo}

Aquí va el texto de la primera sección del segundo anexo...

